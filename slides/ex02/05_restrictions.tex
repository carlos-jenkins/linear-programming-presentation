\begin{frame}{Example 02 - Restrictions}

The first restriction is abour the amount of trucks that could be painted by
day. If the painting shop were completely devoted to painting type 1 trucks, 800
per day could be painted, whereas if the painting shop were completely devoted
to painting type 2 trucks, 700 per day could be painted.
So just painting 1 truck type 1 could be expressed as $\frac{1}{800}\colora{x_1}$
and just painting 1 truck type 2 could be expressed as $\frac{1}{700}\colorb{x_2}$.
The same happens with assembly, just assembling 1 truck type 1 could be
represented as $\frac{1}{1500}\colora{x_1}$ and just assembling 1 truck type 2
could be expressed as $\frac{1}{1200}\colorb{x_2}$. These 2 restrictions refers
just to 1 truck, so both are less or equal than 1. The last restriction is the
trivial one, that all variables must be greater or equal than 0.

\end{frame}
