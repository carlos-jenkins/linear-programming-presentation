\begin{frame}{Example 18 - Problem (1)}
A paper recycling plant processes box board, tissue paper, newsprint, and book
paper into pulp that can be used to produce three grades of recycled paper
(grades 1, 2, and 3). The prices per ton and the pulp contents of the four
inputs are shown in the following table.

\begin{center}
\begin{tabular}{lrr}
\hline
  \cellcolor{gray90}\textbf{Input}
& \cellcolor{gray90}\textbf{Cost}
& \cellcolor{gray90}\textbf{Pulp Content} \\
\hline
Box board    &  \$5 & 15\% \\
Tissue paper &  \$6 & 20\% \\
Newsprint    &  \$8 & 30\% \\
Book paper   & \$10 & 40\% \\
\hline
\end{tabular}
\end{center}

\end{frame}

\begin{frame}{Example 18 - Problem (2)}
Two methods, de-inking and asphalt dispersion, can be used to process the four
inputs into pulp. It costs \$20 to de-ink a ton of any input. The process of
de-inking removes 10 percent of the input’s pulp, leaving 90\% of the original
pulp. It costs \$15 to apply asphalt dispersion to a ton of material. The
asphalt dispersion process removes 20\% of the input’s pulp. At most 3000 tons
of input can be run through the asphalt dispersion process or the de-inking
process.
\end{frame}

\begin{frame}{Example 18 - Problem (3)}
Grade 1 paper can only be produced with newsprint or book paper pulp; grade 2
paper only with book paper, tissue paper, or box board pulp. To meet its current
demands, the company needs 500 tons of pulp for grade 1 paper, 500 tons of pulp
for grade 2 paper, and 600 tons of pulp for grade 3 paper. Formulate an LP to
minimize the cost of meeting demands for pulp.
\end{frame}
