\begin{frame}{Example 06 - Restrictions}

The first restriction would be the amount or percentual distribution of each
type of valve que each supplier needs; that is each percentage is divided by
100 and is multiplied by the $x$ that corresponds to the total amount of valves.

We know the we need to buy at least 500 large valves, with 40\% being from
supplier 1, 30\% from supplier 2 and 20\% for supplier 3; that is:
$0.4\colora{x_1} + 0.3 \colorb{x_2} + 0.2\colorc{x_3} \ge 500$. In a similar way
we have some restrictions for medium size and small size valves, which give
$0.4\colora{x_1} + 0.35\colorb{x_2} + 0.2\colorc{x_3} \ge 300$ and
$0.2\colora{x_1} + 0.35\colorb{x_2} + 0.6\colorc{x_3} \ge 300$.

The last restrictions is that we can't buy more than 500 total valves from each
supplier, and obviously we can't buy negative valves. Both restriction can be
modeled like:
$\colora{x_1}, \colorb{x_2}, \colorc{x_3} \le 500$ and
$\colora{x_1}, \colorb{x_2}, \colorc{x_3} \ge 0$.

\end{frame}
