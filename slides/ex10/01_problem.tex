\begin{frame}{Example 10 - Problem (1)}
A company produces six products in the following fashion. Each unit of raw
material purchased yields four units of product 1, two units of product 2, and
one unit of product 3. Up to 1200 units of product 1 can be sold, and up to 300
units of product 2 can be sold. Each unit of product 1 can be sold or processed
further. Each unit of product 1 that is processed yields a unit of product 4.
Demand for products 3 and 4 is unlimited. Each unit of product 2 can be sold or
processed further. Each unit of product 2 that is processed further yields 0.8
unit of product 5 and 0.3 unit of product 6. Up to 1000 units of product 5 can
be sold, and up to 800 units of product 6 can be sold. Up to 3000 units of raw
material can be purchased at \$6 per unit.
\end{frame}

\begin{frame}{Example 10 - Problem (2)}
Leftover units of products 5 and 6 must be destroyed. It costs \$4 to destroy
each leftover unit of product 5 and \$3 to destroy each leftover unit of product
6. Ignoring raw material purchase costs, the per-unit sales price and production
costs for each product are shown in the following table. Formulate an LP whose
solution will yield a profit-maximizing production schedule.

\begin{center}
\begin{tabular}{lrrr}
\hline
  \cellcolor{gray90}\textbf{Product}
& \cellcolor{gray90}\textbf{Sales price}
& \cellcolor{gray90}\textbf{Production cost} \\
\hline
Product 1 &  \$7 & \$4 \\
Product 2 &  \$6 & \$4 \\
Product 3 &  \$4 & \$2 \\
Product 4 &  \$3 & \$1 \\
Product 5 & \$20 & \$5 \\
Product 6 & \$35 & \$5 \\
\hline
\end{tabular}
\end{center}

\end{frame}
