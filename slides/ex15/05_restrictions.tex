\begin{frame}{Example 15 - Restrictions (1)}

The first restriction is given by the hours that are required for each product
in the machine 1. Knowing that there are 5 machines of this type available and
that the factory is open 40 hours, therefore with the hours given in the board
of the machine 1 we have that 2
$2\colora{X_1} +  3\colorb{X_2} +  4\colorc{X_3} \le 200$. Similarly a
restriction of the same nature is obtained for the other remaining two machines:
for machine 2 we have $3\colora{X_1} +  5\colorb{X_2} +  6\colorc{X_3} \le 120$
for there are 3 machines available. For the machine 3 we have
$4\colora{X_1} +  7\colorb{X_2} +  9\colorc{X_3} \le 160$
for there are 4 machines available.

\end{frame}


\begin{frame}{Example 15 - Restrictions (2)}

There is also the restriction of each worker working a total of 35 hours per
week for a total of 10 workers and when each product is in a machine it must be
handled by a worker, because of this all the hours required by product are
added, so for product 1 9 hours are required; for product 2, 15 hours and for
product 3 19 hours from what we obtain:
$9\colora{X_1} + 15\colorb{X_2} + 19\colorc{X_3} \le 350$. Finally there is the
trivial restriction that no negative products can exits.

\end{frame}
